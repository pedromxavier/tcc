%%
%% This is file `example.tex',
%% generated with the docstrip utility.
%%
%% The original source files were:
%%
%% coppe.dtx  (with options: `example')
%% 
%% This is a sample monograph which illustrates the use of `coppe' document
%% class and `coppe-unsrt' BibTeX style.
%% 
%% \CheckSum{1416}
%% \CharacterTable
%%  {Upper-case    \A\B\C\D\E\F\G\H\I\J\K\L\M\N\O\P\Q\R\S\T\U\V\W\X\Y\Z
%%   Lower-case    \a\b\c\d\e\f\g\h\i\j\k\l\m\n\o\p\q\r\s\t\u\v\w\x\y\z
%%   Digits        \0\1\2\3\4\5\6\7\8\9
%%   Exclamation   \!     Double quote  \"     Hash (number) \#
%%   Dollar        \$     Percent       \%     Ampersand     \&
%%   Acute accent  \'     Left paren    \(     Right paren   \)
%%   Asterisk      \*     Plus          \+     Comma         \,
%%   Minus         \-     Point         \.     Solidus       \/
%%   Colon         \:     Semicolon     \;     Less than     \<
%%   Equals        \=     Greater than  \>     Question mark \?
%%   Commercial at \@     Left bracket  \[     Backslash     \\
%%   Right bracket \]     Circumflex    \^     Underscore    \_
%%   Grave accent  \`     Left brace    \{     Vertical bar  \|
%%   Right brace   \}     Tilde         \~}
%%
\documentclass[grad,numbers]{coppe}
\usepackage{amsmath,amssymb}
\usepackage{hyperref}
\usepackage[utf8]{inputenc}
\usepackage[brazil]{babel}
\usepackage[T1]{fontenc}

\usepackage{graphicx}
\usepackage{tikz}
\usepackage{float}

\makelosymbols
\makeloabbreviations

\begin{document}
  \title{Satyrus III: Compilador para Computador Quântico}
  \foreigntitle{Satyrus III: Compiler for Quantum Computer}
  \author{Pedro}{Maciel Xavier}
  \advisor{Prof.}{Priscila}{Machado Vieira Lima}{Ph.D.}
  \advisor{Prof.}{Felipe}{Maia Galvão França}{Ph.D.}
  %\advisor{Prof.}{Nome do Terceiro Orientador}{Sobrenome}{D.Sc.}

  \examiner{Prof.}{Nome do Primeiro Examinador Sobrenome}{D.Sc.}
  \examiner{Prof.}{Nome do Segundo Examinador Sobrenome}{Ph.D.}
  \examiner{Prof.}{Nome do Terceiro Examinador Sobrenome}{D.Sc.}
  \examiner{Prof.}{Nome do Quarto Examinador Sobrenome}{Ph.D.}
  \examiner{Prof.}{Nome do Quinto Examinador Sobrenome}{Ph.D.}
  
  \department{ECI}% Confira a tabela a seguir para saber como preencher o comando \department de acordo com seu curso (Graduação - Poli) ou programa (Pós-Graduação - COPPE).
  
  %%%%%% Para alunos da POLI %%%%%%
  
  %% Course											Option
  %% Engenharia Ambiental                             EA
  %% Engenharia Civil                                 ECV
  %% Engenharia de Computação e Informação            ECI
  %% Engenharia de Controle e Automação               ECA
  %% Engenharia de Materiais                          EMAT
  %% Engenharia de Petróleo                           EPT
  %% Engenharia de Produção                           EPR
  %% Engenharia Eletrônica e de Computação            EEC
  %% Engenharia Elétrica                              EET
  %% Engenharia Mecânica                              EMC
  %% Engenharia Metalúrgica                           EMET
  %% Engenharia Naval e Oceânica                      ENO
  %% Engenharia Nuclear                               ENU
  
  
  %%%%%% Para alunos da COPPE %%%%%%
  
  %% Program											Option
  %% Engenharia Biomédica								PEB
  %% Engenharia Civil									PEC
  %% Engenharia Elétrica								PEE
  %% Engenharia Mecânica								PEM
  %% Engenharia Metalúrgica e de Materiais				PEMM
  %% Engenharia Nuclear									PEN
  %% Engenharia Oceânica								PENO
  %% Planejamento Energético							PPE
  %% Engenharia de Produção								PEP
  %% Engenharia Química									PEQ
  %% Engenharia de Sistemas e Computação				PESC
  %% Engenharia de Transportes							PET
  
  
  \date{12}{2021}

  \keyword{Compiladores}
  \keyword{Otimização}
  \keyword{Computação Quântica}
  \keyword{Lógica}

  \maketitle

  \frontmatter
  
  \makecatalog
  
  \dedication{A alguém cujo valor é digno desta dedicatória.}

  \chapter*{Agradecimentos}

  Gostaria de agradecer a todos.

  \begin{abstract}

  Apresenta-se, nesta tese, ...

  \end{abstract}

  \begin{foreignabstract}

  In this work, we present ...

  \end{foreignabstract}

  \tableofcontents
  \listoffigures
  \listoftables
  \printlosymbols
  \printloabbreviations

  \mainmatter
%  \doublespacing
  % Introdução
  \chapter{Introdução}
  
  Um compilador é um programa que transforma o código de um programa em um outro código, numa linguagem potencialmente diferente da linguagem de entrada\cite{aho:86}. O caso de uso mais comum se dá entre linguagens como \textit{C} e \textit{Fortran} que são traduzidas para o \textit{Assembly}, permitindo expressar instruções de máquina dando como entrada expressões de mais alto nível, ou seja, mais próximas da linguagem natural. Uma outra aplicação recorrente para os compiladores é a otimização de programas. Neste caso, a saída pode estar escrita na mesma linguagem que a entrada, representando o mesmo programa, mas com uma sequência de instruções mais eficiente que o original. \par
  \begin{figure}[H]
    \centering
    \begin{tikzpicture}[nodes={fill=gray!20},
    row sep=0.3cm,column sep=0.5cm]
        \node[rectangle] (input) at (0,0) {Programa fonte};
        \node[rectangle, fill=blue!30] (compiler) at (4, 0) {Compilador};
        \draw[->] (input) to[out=0,in=180] (compiler);
        \node[rectangle] (target) at (8, 0) {Programa alvo};
        \draw[->] (compiler) to[out=0,in=180] (target);
        \node[rectangle, fill=red!30] (error) at (4, -2) {Mensagens de erro};
        \draw[->] (compiler) to (error);
    \end{tikzpicture}
    \label{fig:compiler-flow}
    \caption{A estrutura básica de um compilador.}
\end{figure}

  % Revisão Bibliográfica
  \input{projeto-2.tex}
  
  % Metodologia
  \chapter{Metodologia}

    \begin{figure}[H]
    %%
    %
    \def\w{3.2cm}
    \def\s{0.3cm}
    \def\h{1.0cm}

    \def\stack#1#2#3{\node[rectangle] (input) at (\w * #1 + \s * #1, \h * #2) {#3};}
    \def\xtack#1#2#3{\node[rectangle, fill=none, draw=none] (input) at (\w * #1 + \s * #1, \h * #2) {};}
    \def\arrow#1{\draw[->] (\w * #1 + \s * #1 + \w * 0.5, 0) -- (\w * #1 + \s * #1 + \w * 0.5 + \s, 0);}
    \def\base#1#2{\node[rectangle, fill=#2!30, minimum height=0.2cm] (input) at (\w * #1 + \s * #1, -\h+0.25cm) {};}
    %
    %%
    \centering
    \begin{tikzpicture}[nodes={draw=black!90, fill=gray!10,
        minimum height=\h,
        minimum width=\w,
        },
    row sep=0.3cm,column sep=0.5cm]
        \base{0}{blue}
        \stack{0}{3}{$\forall i\in \Omega_i : f_i$}
        \stack{0}{2}{$\forall j \in \Omega_j : f_{i, j}$}
        \stack{0}{1}{$\exists k \in \Omega_k$}
        \stack{0}{0}{$\varphi_{i, j, k}$}
        %\arrow{0}
        \base{1}{red}
        %\stack{1}{3}{$\forall i\in \Omega_i$}
        %\stack{1}{2}{$\forall j \in \Omega_j$}
        %\stack{1}{1}{$\exists k \in \Omega_k$}
        %\stack{1}{0}{$\Phi_{i, j, k}$}
        %\arrow{1}
    \end{tikzpicture}
    \bigskip\break
    \begin{tikzpicture}[nodes={draw=black!90, fill=gray!5,
        minimum height=\h,
        minimum width=\w,
        },
    row sep=0.3cm,column sep=0.5cm]
        \base{0}{blue}
        \xtack{0}{3}{$\forall i\in \Omega_i$}
        \stack{0}{2}{$\forall j \in \Omega_j$}
        \stack{0}{1}{$\exists k \in \Omega_k$}
        \stack{0}{0}{$\varphi_{i, j, k}$}
        %\arrow{0}
        \base{1}{red}
        \stack{1}{0}{$\Pi_{i}, i\in \Omega_i$}
        %\stack{1}{2}{$\forall j \in \Omega_j$}
        %\stack{1}{1}{$\exists k \in \Omega_k$}
        %\stack{1}{0}{$\Phi_{i, j, k}$}
        %\arrow{1}
    \end{tikzpicture}
    %%
    \label{fig:compiler-stack}
    \caption{O processo de análise dos quantificadores.}
\end{figure}

    \section{WTA}

    \input{figures/compiler-wta.tex}
  
  
  
  \chapter{Resultados e Discussões}
  
  
  
  \chapter{Conclusões}
  
  \backmatter
  \bibliographystyle{coppe-unsrt}
  \bibliography{bibliografia}

  \appendix
  \chapter{Algumas Demonstrações}
\end{document}
%% 
%%
%% End of file `example.tex'.
